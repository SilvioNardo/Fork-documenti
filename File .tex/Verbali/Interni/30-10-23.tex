\documentclass[12pt,a4paper]{article}
\usepackage[italian, english]{babel}
\usepackage[utf8]{inputenc}
\usepackage[T1]{fontenc}
\usepackage{amsmath}
\usepackage{amsfonts}
\usepackage{xurl}
\usepackage{graphicx}
\usepackage[dvipsnames]{xcolor}
\usepackage{tikz}\usepackage{multirow}
\usepackage{comment}
\usepackage{tabularx}
\usepackage{multirow}

\makeatletter
\newcommand*{\rom}[1]{\expandafter\@slowromancap\romannumeral #1@}
\makeatother

%=====================INIZIO DOCUMENTO=====================
\begin{document}

%%%%%%%%%%%%%%%% header %%%%%%%%%%%%%%%%%%%%%%%%%%%%%%

\noindent\begin{minipage}{0.3\textwidth}
    \includegraphics[width=\linewidth]{logo.png}
\end{minipage}%
\hfill%
\begin{minipage}{0.6\textwidth}\raggedright
    \huge
    ERROR\_418\\
    Verbale 18/10/23
\end{minipage}

%%%%%%%%%%%%%%%% Sezione informativa %%%%%%%%%%%%%%%%%%
\large
\setlength{\extrarowheight}{9pt}
\raggedright
\begin{tabularx}{0.9\textwidth} [right] {
        >{\raggedright\arraybackslash}X
        >{\raggedright\arraybackslash}X
    }
    Mail:           & error418swe@gmail.com                              \\
    Redattori:      & Riccardo Carraro, Antonio Oseliero                 \\
    Verificatori:   & Alessio Banzato, Giovanni Gardin, Rosario Zaccone \\
    Amministratori: & Silvio Nardo, Mattia Todesco                       \\
    Destinatari:    & T. Vardanega, R. Cardin
\end{tabularx}
%%%%%%%%%%%%%%%%%%%%%% Presenze %%%%%%%%%%%%%%%%%%%%%%%%
\vspace{3mm}\hline\hline
\raggedright
\begin{tabular}{c c}
    \multicolumn{2}{c}{Inizio Meeting: 14:30 \hspace{4mm}
    Fine Meeting: 18:30 \hspace{4mm} Durata: 4:00h} \\
    Presenze: &                                    \\
\end{tabular}

\begin{center}
    \begin{tabular}{ |c|c|c|c|c| }
        \hline
        Nome     & Durata Presenza &  & Nome     & Durata Presenza \\
        \hline
        Antonio  & 4:00h           &  & Alessio  & 4:00h           \\
        \hline
        Riccardo & 4:00h           &  & Giovanni & 4:00h           \\
        \hline
        Rosario  & 4:00h           &  & Silvio   & 4:00h           \\
        \hline
        Mattia   & 4:00h           &  &          &                 \\
        \hline
    \end{tabular}
\end{center}
\newpage

Ordine del giorno:
\begin{itemize}
    \item Analisi carico di lavoro per la stesura del Preventivo Costi;
    \item Redazione del preventivo dei costi;
    \item Finalizzazione redazione Lettera di presentazione;
    \item Finalizzazione redazione Valutazione dei Capitolati;
    \item Organizzazione gruppi di lavoro per la revisione finale dei documenti, quali:
        \begin{itemize}
            \item Verbali interni;
            \item Verbali esterni;
            \item Valutazione dei Capitolati;
            \item Lettera di presentazione.
        \end{itemize}
    \item Riorganizzazione repo.
\end{itemize}

A fini organizzativi, il team, a seguito dell'analisi congiunta riguardo il carico di lavoro e alla revisione della lettera di presentazione, si è diviso in sotto gruppi per lavorare in contemporanea su documenti e aspetti diversi.\\
I gruppi formatisi prevedono:
\begin{itemize}
    \item Primo sotto gruppo: 3 membri -> revisione Valutazione capitolati;
    \item Secondo sotto gruppo: 2 membri -> revisione Capitolati 17,18,21 ottobre;
    \item Terzo sotto gruppo: 2 membri -> revisione Capitolati 25,26,29 ottobre.
\end{itemize}

\section{Analisi carico di lavoro}
Al fine di redigere il preventivo dei costi, è stata necessaria un'accurata analisi del carico di lavoro e del monte ore necessario. L'idea alla base dell'organizzazione del lavoro è stata la suddivisione del periodo di sviluppo (Novembre - Marzo) in tre sotto-periodi che affronteranno aspetti diversi ma fondamentali del prodotto, quali:
\begin{itemize}
    \item Periodo di Analisi dei Requisiti;
    \item Periodo per PoC e RTB;
    \item Periodo per PB.
\end{itemize}

Costi e ore sono stati stabiliti mediante la realizzazione di una tabella su Google Sheet in modo collaborativo con tutti i membri del gruppo.  

\section{Redazione documenti}
\subsection{Preventivo Costi}
A seguito dell'analisi precedente e delle decisioni riguardo ore e costi, si è redatto il documento relativo al Preventivo dei Costi.

\subsection{Lettera di presentazione}
Dopo una revisione collettiva, si è proseguito con la finalizzazione della Lettera di presentazione, con l'aggiunta del link alla repo github dei documenti (link al ramo main) e aggiunta del prezzo preventivato.

\subsection{Valutazione Capitolati}
Uno dei sotto gruppi formati si è occupato della revisione, riorganizzazione e finalizzazione del documento relativo alla valutazione complessiva dei nove capitolati. Il documento, risultando corposo e denso di analisi, ha richiesto una discreta quantità di tempo per essere redatto e validato, prestando attenzione ad includere le risposte ricevute da tutte e nove le aziende proponenti.

\subsection{Revisione Documenti}
I due sotto gruppi impegnati nella revisione dei verbali hanno svolto un completo lavoro di revisione e analisi di correttezza dei verbali fino a quel momento redatti. Questa fase, a causa di un mancato salvataggio dei file .tex di latex, ha purtroppo visto una nuova stesura completa di alcuni verbali anche per piccole correzioni. Le nuove versioni dei verbali rimangono pur sempre una fedele ritrascrizione dei verbali originali, senza modificarne il contenuto, ma migliorandone la stesura, rimuovendo eventuali errori ortografici e non.\\
Terminata la revisione e la redazione dei verbali, si è susseguita un'ulteriore revisione di conferma della documentazione rimanente, come la lettera di presentazione e la valutazione dei capitolati (documento terminato durante lo svolgersi di queste revisioni).

\subsection{Riorganizzazione Repo}
Terminata la redazione dei documenti e la loro revisione il gruppo ha potenzialmente tutto il necessario per procedere alla candidatura. In questo modo, possedendo ora i file ufficiali e revisionati che saranno oggetto di consegna, è stato possibile riorganizzare la repo al fine di eliminare evetuali branch secondari e concludere le pull request, in modo da riunire tutti i documenti prodotti dai vari branch creati al ramo Develop, in attesa di unirli successivamente al ramo Main.

\end{document}