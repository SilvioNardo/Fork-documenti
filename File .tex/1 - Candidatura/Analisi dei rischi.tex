\documentclass[a4paper, twoside]{article}
%% Language and font encodings
\usepackage[english]{babel}
\usepackage[utf8x]{inputenc}
\usepackage[T1]{fontenc}
%% Sets page size and margins
\usepackage[a4paper,top=3cm,bottom=2cm,left=3cm,right=3cm,marginparwidth=1.75cm]{geometry}
%% Useful packages
\usepackage{amsmath}
\usepackage{graphicx}
\usepackage[colorinlistoftodos]{todonotes}
\usepackage[colorlinks=true, allcolors=blue]{hyperref}
\usepackage{tabularx}
\usepackage{booktabs}
\usepackage{caption}
%------------------------
\title{\Huge Analisi dei rischi}
\author{Error\_418}
\newcommand{\HRule}{\rule{\linewidth}{0.5mm}}
\setlength {\marginparwidth }{2cm}
%%page number
\pagestyle{plain}
%---------------
\begin{document}
\sffamily
\begin{titlepage}
%----------------------------------------------------------------------------------------
%	LOGO SECTION
%----------------------------------------------------------------------------------------
\centering
\includegraphics[width=8cm]{logo.png}\\[1.5cm]
%----------------------------------------------------------------------------------------
\center % Center everything on the page
%----------------------------------------------------------------------------------------
%	HEADING SECTIONS
%----------------------------------------------------------------------------------------
\textsf{\Large ERROR\_418}\\[0.5cm]
\textsf{\Large DOCUMENTAZIONE PROGETTO}\\[0.5cm]
%----------------------------------------------------------------------------------------
%	TITLE SECTION
%----------------------------------------------------------------------------------------
\makeatletter
\HRule \\[0.4cm]
{ \huge \bfseries \@title}\\[0.4cm]
\HRule \\[1.5cm]
%----------------------------------------------------------------------------------------
%	AUTHOR SECTION
%----------------------------------------------------------------------------------------

\begin{center} % Centro delle sezioni Redattore e Validatori
    \Large
    \setlength{\extrarowheight}{9pt}
    \raggedright
    \begin{tabularx}{0.9\textwidth} [right] {
            >{\raggedright\arraybackslash}X
            >{\raggedright\arraybackslash}X
        }
        Mail:           & error418swe@gmail.com                              \\
        Redattori:      & Antonio Oseliero, Alessio Banzato                  \\
        Verificatori:   & Riccardo Carraro, Giovanni Gardin, Rosario Zaccone \\
        Amministratori: & Silvio Nardo, Mattia Todesco                       \\
        Destinatari:    & T. Vardanega, R. Cardin
    \end{tabularx}
\end{center}

\vfill % Fill the rest of the page with whitespace
\end{titlepage}
\large

\newpage % Vai alla pagina successiva


%%%%%%%%%%%%%%% SEZIONE CONTENUTO %%%%%%%%%%%%%%%%%%%%%%%%%%%%%%%%%%%%%

\section{Analisi dei rischi}

A ciascun rischio individuato si associano:
\begin{itemize}
    \item Impatto: può essere lieve, medio, grave. Esprime l'effetto generato dall'evento;
    \item Probabilità: da 1 a 5. Esprime la probabilità del verificarsi del rischio;
    \item Conseguenze: effetti collaterali a breve o medio termine che il rischio può comportare.
\end{itemize}

\section{Rischi}

\subsection{Comunicazione con il proponente}

I contatti con il proponente potrebbero subire variazioni nella qualità e nella frequenza a causa di problematiche fuori dal controllo del gruppo. Questa situazione potrebbe causare un rallentamento significativo del lavoro, soprattutto durante l'analisi dei requisiti.
\begin{itemize}
    \item Impatto: grave;

    \item Probabilità: 1;

    \item Conseguenze: lo sviluppo potrebbe allontanarsi dalle linee guida o dalle aspettative del proponente, non rispettando quanto preventivato o pianificato. Tale rischio, comporterebbe dunque la produzione di un software non in linea con le richieste conducendo a perdite di tempo per analisi, progettazione e implementazione aggiuntive;

    \item Mitigazione:
        \begin{itemize}
            \item Pianificazione anticipata degli incontri di revisione dell'avanzamento;
            \item Uso di strumenti asincroni per facilitare lo scambio di informazioni tra gruppo e proponente;
            \item Programmazione di incontri periodici di aggiornamento, anche brevi.
        \end{itemize}
\end{itemize}

\subsection{"Effetto sottomarino"}

Uno o più membri potrebbero, per motivi diversi, cessare la partecipazione attiva alle attività del gruppo. È necessario evitare che la durata di queste assenze impedisca il regolare svolgimento delle attività di progetto.

\begin{itemize}
    \item Impatto: medio;

    \item Probabilità: 3;

    \item Conseguenze: i partecipanti che si dovessero trovare in questa situazione rischierebbero di accentuare eventuali incomprensioni nel proprio lavoro senza la possibilità di confrontarsi con gli altri accorgendosi degli errori troppo tardi;

    \item Mitigazione:
        \begin{itemize}
            \item Mantenimento di un dialogo costante sulle problematiche interne al gruppo;
            \item Segnalazione responsabile e preventiva di difficoltà o impedimenti da parte dei singoli membri.
        \end{itemize}
\end{itemize}


\subsection{Rallentamento delle attività}
Tra le difficoltà principali durante lo sviluppo del progetto è la congiunzione tra gli impegni individuali e progettuali. Tale rischio può comportare un rallentamento nel completamento di attività e task assegnate comportando un generale ritardo nello sviluppo.

\begin{itemize}
    \item Impatto: grave;

    \item Probabilità: 4 (\textit{Probabilità aumentata nel periodo della sessione invernale});

    \item Conseguenze: attività non svolte o completate parzialmente determinerebbero uno slittamento della data di consegna e delle scadenze intermedie;
    \item Mitigazione:
        \begin{itemize}
            \item Organizzazione e suddivisione del monte ore con occhio di riguardo a precise date e scadenze;
            \item Incontri e comunicazione costante con i membri del gruppo al fine di rendere note eventuali indisponibilità o impegni;
            \item Uso di strumenti asincroni al fine di permettere a tutti i membri un'equa divisione del lavoro da svolgere nei momenti a loro più comodi, a patto di rispettare le linee guida del Way Of Working.
        \end{itemize}
\end{itemize}


\subsection{Utilizzo delle tecnologie}

Le tecnologie individuate o suggerite durante i processi di analisi e progettazione potrebbero risultare complesse da comprendere e/o integrare.

\begin{itemize}
    \item Impatto: medio;

    \item Probabilità: 4;

    \item Conseguenze: rallentamenti non preventivati che possono avere conseguenze a cascata sulle attività dipendenti;
    \item Mitigazione:
        \begin{itemize}
            \item Accurata pianificazione e stesura delle norme di progetto e Way Of Working;
            \item Assicurarsi che ad ogni membro del gruppo sia chiaro il funzionamento delle tecnologie e delle norme concordate.
        \end{itemize}
\end{itemize}

\end{document}