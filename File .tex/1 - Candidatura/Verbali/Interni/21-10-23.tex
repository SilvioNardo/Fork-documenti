\documentclass[12pt,a4paper]{article}
\usepackage[italian, english]{babel}
\usepackage[utf8]{inputenc}
\usepackage[T1]{fontenc}
\usepackage{amsmath}
\usepackage{amsfonts}
\usepackage{xurl}
\usepackage{graphicx}
\usepackage[dvipsnames]{xcolor}
\usepackage{tikz}\usepackage{multirow}
\usepackage{comment}
\usepackage{tabularx}
\usepackage{multirow}

\makeatletter
\newcommand*{\rom}[1]{\expandafter\@slowromancap\romannumeral #1@}
\makeatother

%=====================INIZIO DOCUMENTO=====================
\begin{document}

%%%%%%%%%%%%%%%% header %%%%%%%%%%%%%%%%%%%%%%%%%%%%%%

\noindent\begin{minipage}{0.3\textwidth}
    \includegraphics[width=\linewidth]{logo.png}
\end{minipage}%
\hfill%
\begin{minipage}{0.6\textwidth}\raggedright
    \huge
    ERROR\_418\\
    Verbale 21/10/23
\end{minipage}

%%%%%%%%%%%%%%%% Sezione informativa %%%%%%%%%%%%%%%%%%
\large
\setlength{\extrarowheight}{9pt}
\raggedright
\begin{tabularx}{0.9\textwidth} [right] {
        >{\raggedright\arraybackslash}X
        >{\raggedright\arraybackslash}X
    }
    Mail:           & error418swe@gmail.com                              \\
    Redattori:      & Antonio Oseliero, Alessio Banzato                  \\
    Verificatori:   & Riccardo Carraro, Giovanni Gardin, Rosario Zaccone \\
    Amministratori: & Silvio Nardo, Mattia Todesco                       \\
    Destinatari:    & T. Vardanega, R. Cardin
\end{tabularx}
%%%%%%%%%%%%%%%%%%%%%% Presenze %%%%%%%%%%%%%%%%%%%%%%%%
\vspace{3mm}\hline\hline
\raggedright
\begin{tabular}{c c}
    \multicolumn{2}{c}{Inizio Meeting: 08:00 \hspace{4mm}
    Fine Meeting: 09:15 \hspace{4mm} Durata:1:15h} \\
    Presenze: &                                    \\
\end{tabular}

\begin{center}
    \begin{tabular}{ |c|c|c|c|c| }
        \hline
        Nome     & Durata Presenza &  & Nome     & Durata Presenza \\
        \hline
        Antonio  & 1:15h           &  & Alessio  & 1:15h           \\
        \hline
        Riccardo & 1:15h           &  & Giovanni & 1:15h           \\
        \hline
        Rosario  & 1:10h           &  & Silvio   & 1:15h           \\
        \hline
        Mattia   & 0:00h           &  &          &                 \\
        \hline

    \end{tabular}
\end{center}

\newpage
Ordine del giorno:

\begin{itemize}
    \item Discussione introduttiva al metodo di lavoro;    
    \item Presentazione lavoro svolto su configurazione ambiente git;
    \item Presentazione lavoro svolto su template della documentazione;
    \item Scelta finale del logo.

\end{itemize}

\section{GitHub}
Antonio ha creato una repository su GitHub e ha impostato Project, issue , tag e milestone sulla stessa, andando poi a dare spiegazione dei principali rami creati al suo interno. \\
Si è deciso di utilizzare i feature branch per il lato implementativo.\\

\section{Documenti}
Creazione e ottimizzazione del template per i verbali del meeting. \\
Creazione delle prime issues su GitHub\\
Decisione di adottare Overleaf per il live editing dei fil .tex.

\section{Way of working}
La prima versione del documento va presentata entro il 31/10/2023 per la candidatura.\\
Spunti di ragionamento a tal proposito:

\begin{itemize}
    \item Cicli di una settimana;    
    \item Meeting periodici;
    \item Feature branch con main protetto;
    \item Test Driven Development;
    \item Continuous Integration con GitHub Actions;
    \item Continuous Deployment con Docker.
\end{itemize}

\section{Meeting aziendale}

Scrivere e-mail alle aziende per concordare un meeting, concentrandosi sull'analisi dei requisiti e su eventuali domande specifiche.\\
Creazione di file condiviso su GitHub per scrivere ognuno le domande da porgere alle aziende.\\
Fissato il compito individuale di trovare 1-2 domande da porre per ogni capitolato entro il 22/10/2023.

\section{Logo}
Logo deciso.

\section{Obiettivi}
\begin{itemize}
    \item Caricare logo su Github;     
    \item (Entro il 22/10/2023) raccogliere domande per le aziende;
    \item (Entro il 31/10/2023) preparazione Norme di progetto e lettera di presentazione con disponibilità oraria;
    \item Caricamento di due file su GitHub, uno per le domande da rivolgere ai proponenti e l'altro con una bozza del Way of Working.
\end{itemize}

\end{document}