\documentclass[a4paper, twoside]{article}
%% Language and font encodings
\usepackage[english]{babel}
\usepackage[utf8x]{inputenc}
\usepackage[T1]{fontenc}
%% Sets page size and margins
\usepackage[a4paper,top=3cm,bottom=2cm,left=3cm,right=3cm,marginparwidth=1cm]{geometry}
%% Useful packages
\usepackage{amsmath}
\usepackage{graphicx}
\usepackage[colorinlistoftodos]{todonotes}
\usepackage[colorlinks=false, allcolors=black]{hyperref}
\usepackage{tabularx}
\usepackage{booktabs}
\usepackage{caption}
%------------------------
\title{\Huge Valutazione Capitolati}
\author{Error\_418}
\newcommand{\HRule}{\rule{\linewidth}{0.5mm}}
\setlength {\marginparwidth }{2cm}
%%page number
\pagestyle{plain}
%---------------
\begin{document}
\sffamily
\begin{titlepage}
%----------------------------------------------------------------------------------------
%	LOGO SECTION
%----------------------------------------------------------------------------------------
\centering
\includegraphics[width=8cm]{logo.png}\\[1.5cm]
%----------------------------------------------------------------------------------------
\center % Center everything on the page
%----------------------------------------------------------------------------------------
%	HEADING SECTIONS
%----------------------------------------------------------------------------------------
\textsf{\Large ERROR\_418}\\[0.5cm]
\textsf{\Large DOCUMENTAZIONE PROGETTO}\\[0.5cm]
%----------------------------------------------------------------------------------------
%	TITLE SECTION
%----------------------------------------------------------------------------------------
\makeatletter
\HRule \\[0.4cm]
{ \huge \bfseries \@title}\\[0.4cm]
\HRule \\[1.5cm]
%----------------------------------------------------------------------------------------
%	AUTHOR SECTION
%----------------------------------------------------------------------------------------

\begin{center} % Centro delle sezioni Redattore e Validatori
    \Large
    \setlength{\extrarowheight}{9pt}
    \raggedright
    \begin{tabularx}{0.9\textwidth} [right] {
            >{\raggedright\arraybackslash}X
            >{\raggedright\arraybackslash}X
        }
        Mail:           & error418swe@gmail.com                              \\
        Redattori:      & Antonio Oseliero, Alessio Banzato                  \\
        Verificatori:   & Riccardo Carraro, Giovanni Gardin, Rosario Zaccone \\
        Amministratori: & Silvio Nardo, Mattia Todesco                       \\
        Destinatari:    & T. Vardanega, R. Cardin
    \end{tabularx}
\end{center}

\vfill % Fill the rest of the page with whitespace
\end{titlepage}
\large

\newpage % Vai alla pagina successiva

\tableofcontents
\newpage

%%%%%%%%%%%%%%% SEZIONE CONTENUTO %%%%%%%%%%%%%%%%%%%%%%%%%%%%%%%%%%%%%


\section{Valutazione Capitolato Scelto, C5 -- WMS3: Warehouse Management 3D}
    \subsection{Descrizione}
        \begin{itemize}
            \item Proponente:
            \begin{itemize}
                \item Sanmarco Informatica S.p.a.
            \end{itemize}
            \item Committente:
            \begin{itemize}
                \item Prof. Tullio Vardanega e Prof.
                Riccardo Cardin.
            \end{itemize}
            \item Obbiettivo:
            \begin{itemize}
                \item Creare un ambiente 3D che permetta la gestione
                di un magazzino.
            \end{itemize}
        \end{itemize}
        Lo scopo di questo capitolato è quello di fornire un
        modo nuovo per visualizzare un magazzino tramite il
        3D, controllare la posizione di materiali e scaffalature con
        la possibilità di modificarne la posizione.
        Il proponente richiede che il programma permetta sessioni
        volatili senza persistenza delle modifiche (lo spostamento di
        un elemento nel magazzino viene gestito tramite una notifica
        inviata ai magazzinieri che noi applichiamo sotto forma di API
        in modo che sia poi il richiedente ad integrarla con i suoi
        sistemi).
    \subsection{Dominio Tecnologico}
        L'azienda (all'interno del capitolato e nel meeting privato
        con il nostro gruppo) lascia molta libertà in merito alle
        tecnologie da utilizzare. Per quanto riguarda lo sviluppo
        dell'ambiente 3D suggerisce:
        \begin{itemize}
            \item Three.js;
            \item Unity;
            \item Unreal.
        \end{itemize}
        Anche per quanto riguarda la scelta del database l'azienda
        lascia libertà decisionale suggerendo comunque l'uso di un
        dabase relazionale.

    \subsection{Motivazione della Scelta}
        \begin{itemize}
            \item L'idea è molto originale, utile e permette di
            esplorare ambiti poco conosciuti come la modellazione 3D;
            \item La libreria Three.js risulta molto interessante e moderna,
            tenendo conto anche del supporto che l'azienda può fornire in quanto
            specializzata su di essa;
            \item L'azienda è sembrata molto disponibile, come mostratosi dalle
            risposte tempestive dateci negli ultimi giorni, difatti metterà a
            disposizione figure di diverso livello in modo tale da poter
            rispondere nella maniera più appropriata alle nostre esigenze, come
            detto durante la giornata di presentazione dei capitolati.
        \end{itemize}

    \subsection{Conclusioni}
        Nonostante i dubbi iniziali, la proposta offerta da Sanmarco Informatica
        S.p.a. ha cominciato sempre di più ad interessarci, complice anche la
        disponibilità, simpatia e competenza del referente che ci ha offerto
        subito l'opportunità di un incontro per chiarire i punti non chiari del
        capitolato. La possibilità di lavorare con un ambiente 3D è stimolante e
        sono chiare le possibili applicazioni reali.

        \section{Valutazione C1 -- Knowledge Management AI}
        \subsection{Descrizione}
            \begin{itemize}
                \item Proponente:
                \begin{itemize}
                    \item azzurrodigitale.
                \end{itemize}
                \item Committente:
                \begin{itemize}
                    \item Prof. Tullio Vardanega e Prof.
                    Riccardo Cardin.
                \end{itemize}
                \item Obbiettivo:
                \begin{itemize}
                    \item Semplificare la consultazione di informazioni all'interno di un'azienda.
                \end{itemize}
            \end{itemize}
            Il capitolato punta a semplificare la consultazione di
            diverse tipologie di informazioni aziendali sfruttando
            un'intelligenza artificiale, il cui training avverrà
            grazie API di terze parti. In particolare si richiede
            lo sviluppo di una piattaforma web all'interno della quale sarà possibile caricare, consultare ed eliminare i documenti (che verranno poi indicizzati) e utilizzare una chat per interagire con il motore di intelligenza artificiale.
        \subsection{Dominio Tecnologico}
            L'azienda consiglia alcune tecnologie sia per la parte legata alla creazione della piattaforma web che per la parte di elaborazione dei documenti e API per l'intelligenza artificiale. Le tecnologie consigliate sono le seguenti:
            \begin{itemize}
                \item Node.js;
                \item Angular;
                \item OpenAI;
                \item LangChain.
            \end{itemize}
        \subsection{Conclusioni}
        Il capitolato risulta essere interessante per l'adozione di tecnologie innovative e in particolare per l'ampio raggio d'uso del prodotto che propone, in quanto risulta essere molto utile in diversi ambienti di lavoro. Ad esempio potrebbe essere utilizzato sia in una postazione d'ufficio per richiedere informazioni di carattere amministrativo, che in una postazione all'interno di una fabbrica per comprendere il funzionamento di un determinato strumento.

    \section{Valutazione C2 -- Sistemi di Raccomandazione}
        \subsection{Descrizione}
            \begin{itemize}
                \item Proponente:
                \begin{itemize}
                    \item Ergon Informatica.
                \end{itemize}
                \item Committente:
                \begin{itemize}
                    \item Prof. Tullio Vardanega e Prof.
                    Riccardo Cardin.
                \end{itemize}
                \item Obbiettivo:
                \begin{itemize}
                    \item Utilizzare il machine learning per creare un sistema di raccomandazione.
                \end{itemize}
            \end{itemize}
            In questo capitolato si vuole creare un sistema di raccomandazione che usi il machine learning per migliorare le sue funzionalità. Il prodotto sarà composto da:
            \begin{itemize}
                \item Un database che raccoglie tutti i dati relativi al comportamento dei clienti, e quindi anche i prodotti a loro correlati;
                \item Il sistema di raccomandazione, che utilizza i dati del database;
                \item Un'interfaccia utente che permetta di visualizzare i primi $N$ prodotti correlati a un dato utente oppure i primi $N$ clienti correlati a un dato prodotto.
            \end{itemize}
            Questo sistema dovrà quindi calcolare e stimare le correlazioni tra clienti e prodotti, ma anche tra clienti stessi, in modo da utilizzare queste correlazioni anche per gestire le correlazioni sui prodotti.
        \subsection{Dominio Tecnologico}
        Risulta esserci molta scelta per quanto riguarda le tecnologie, dato che l'azienda ne propone diverse lasciando anche la libertà di adottarne altre.\\
        Per quanto riguarda il database vengono consigliati:
        \begin{itemize}
            \item Sql Server Express;
            \item MySql;
            \item MariaDB.
        \end{itemize}
        Per il sistema di raccomandazione:
        \begin{itemize}
            \item ML.NET;
            \item Surprise (libreria Python).
        \end{itemize}
        Mentre per l'interazione tra database e applicativo vengono consigliate:
        \begin{itemize}
            \item Entity Framework (ORM), in caso si usi ML.NET;
            \item ODBC, in caso si usi Surprise;
            \item Middleware, ad esempio JSON, se si vuole l'indipendenza del sistema dal database.
        \end{itemize}
        L'azienda, inoltre, rende possibile la condivisione di un set di dati da usare per l'apprendimento del modello di machine learning.
        \subsection{Conclusioni}
        La logica alla base del capitolato è molto complessa, e inoltre le capacità del gruppo risultano essere non allineate. Tutto ciò ha fatto pensare che la scelta di questo capitolato avrebbe portato a una situazione in cui il tempo di studio richiesto per la piena comprensione delle tecnologie e del capitolato in sé avrebbe causato un rallentamento considerevole del ritmo di lavoro.

    \section{Valutazione C3 -- Easy Meal}
        \subsection{Descrizione}
            \begin{itemize}
                \item Proponente:
                \begin{itemize}
                    \item Imola Informatica.
                \end{itemize}
                \item Committente:
                \begin{itemize}
                    \item Prof. Tullio Vardanega e Prof.
                    Riccardo Cardin.
                \end{itemize}
                \item Obbiettivo:
                \begin{itemize}
                    \item Web app per migliorare l'esperienza culinaria nei ristoranti.
                \end{itemize}
            \end{itemize}
            Imola Informatica propone lo sviluppo di EasyMeal, una web app innovativa che vuole trasformare il settore dei ristoranti, semplificando la prenotazione e l'esperienza culinaria per gli utenti.\\
            I clienti possono anticipare l'esperienza culinaria creando il proprio ordine da qualsiasi luogo in base alle proprie esigenze, allergie e preferenze alimentari, oltre che specificando l'orario di arrivo nel locale. L'applicazione facilita l'interazione con lo staff del ristorante, consente la divisione del conto tra i partecipanti e contribuisce a ridurre lo spreco alimentare grazie a una pianificazione della spesa più precisa. Includendo funzionalità come la registrazione, la prenotazione di tavoli, l'ordinazione collaborativa dei pasti, l'interazione con il personale del ristorante, la divisione del conto, la consultazione delle prenotazioni da parte dei ristoratori e la possibilità di inserire feedback e recensioni, EasyMeal si propone di offrire una convenienza, personalizzazione ed efficienza superiori sia ai clienti che ai ristoratori.
        \subsection{Dominio Tecnologico}
        Il proponente non fornisce tecnologie particolari e lascia libertà di scelta. L'unico vincolo imposto è che venga sviluppata un'applicazione web responsive (PC, iOS e Android).
        \subsection{Conclusioni}
        Lo sviluppo di un'applicazione sia per iOS che Android risulta essere molto interessante, e sono state apprezzate la precisione con cui sono stati presentati i requisiti e l'alta disponibilità da parte dell'azienda anche per quanto riguarda gli incontri periodici per monitorare l'avanzamento del progetto. Il numero di richieste minime risulta però essere elevato e rischia di portare a tempi di sviluppo molto lunghi.

\section{Valutazione C4 -- A ChatGPT plugin with Nuvolaris}
        \subsection{Descrizione}
        \begin{itemize}
            \item Proponente:
            \begin{itemize}
                \item Nuvolaris.
            \end{itemize}
            \item Committente:
            \begin{itemize}
                \item Prof. Tullio Vardanega e Prof.
                Riccardo Cardin.
            \end{itemize}
            \item Obbiettivo:
            \begin{itemize}
                \item Creare un plugin di ChatGPT usando Nuvolaris serverless.
            \end{itemize}
        \end{itemize}
        Il capitolato si propone di ridurre, tramite IA, la barriera in ingresso
        per la gestione di setup di cloud computing complessi e normalmente
        riservati ad utenti esperti.\\
        Il capitolato prevede lo studio e l'utilizzo di diverse tecnologie
        per ottenere:
        \begin{itemize}
            \item Costruzione e utilizzo di plugin di ChatGPT;
            \item Automatizzazione della costruzione e modifica di applicazioni
            in base alla richiesta dell'utente;
            \item Una serie di template da cui poi verranno generate le
            applicazioni richieste;
            \item Corretta gestione e modifica automatica dei file di
            configurazione per la generazione delle applicazioni.
        \end{itemize}

        \subsection{Dominio Tecnologico}
        ChatGpt, Nuvolaris, Redis, in base al tipo di applicazione possono
        variare le tecnologie utilizzate, ad esempio per applicazioni CRUD
        si può usare SQL.

        \subsection{Conclusioni}
        Il capitolato non risulta chiarissimo nella proposta, sono state necessarie
        delle mail all'azienda per chiarire alcuni punti non chiarissimi che sono
        arrivate molto celermente. In generale uno dei motivi per cui abbiamo
        faticato nel capire la proposta di Nuvolaris è che i membri del team non
        si sono mai trovati a dover interagire con Docker o Kubernetes.\\
        L'azienda mette a disposizione molte risorse quali: account ChatGPT Pro,
        documentazione, ambiente Nuvolaris dedicato e completo in cloud.

\section{Valutazione C6 -- SyncCity: Smart city monitoring platform}
    \subsection{Descrizione}
        \begin{itemize}
            \item Proponente:
            \begin{itemize}
                \item SyncLab.
            \end{itemize}
            \item Committente:
            \begin{itemize}
                \item Prof. Tullio Vardanega e Prof.
                Riccardo Cardin.
            \end{itemize}
            \item Obbiettivo:
            \begin{itemize}
                \item Creare una piattaforma che rappresenti in una serie di
                dashboard dati provenienti da molti sensori per il monitoraggio
                della qualità della vita di una città.
            \end{itemize}
        \end{itemize}
        L'azienda propone lo sviluppo di una piattaforma che rappresenti una serie
        di dati ricavati da diversi sensori collocati in una città (questi dati devono
        essere opportunamente simulati o ottenuti da sensori reali) in modo da rappresentarne
        lo stato di salute. Il capitolato prevede lo studio e l'utilizzo di
        diverse tecnologie per ottenere:
        \begin{itemize}
            \item Implementazione di Simulatori di dati con documentazione
            relativa;
            \item Configurazione del database per lo storage dei dati;
            \item Piattaforma di stream processing mediante invio di dati a
            Kafka;
            \item Sviluppo di una dashboard a fini di consultazione dei dati
            raccolti mediante Grafana;
            \item Testing con copertura >= 80\%.
        \end{itemize}
        Il proponente nomina come informazioni da rappresentare
        per esempio:
        \begin{itemize}
            \item Temperatura, espressa in °C;
            \item Polveri sottili, espressa in $\mu g/mc$;
            \item Umidità, espressa in percentuale;
            \item Livello dell'acqua nella zona d'installazione
            del sensore;
            \item Guasti elettrici, 0 o 1 in caso si verifichi
            un interruzione della corrente nella zona d'installazione
            del sensore;
            \item Riempimento dei vari conferitori di un isola
            ecologica, 0 o 1 a seconda se sia piena o meno.
        \end{itemize}

    \subsection{Dominio Tecnologico}
        L'azienda suggerisce come tecnologie da impiegare:
        \begin{itemize}
            \item Script Pyton (o altri linguaggi) e librerie per la generazione
            dati per ottenere una simulazione dati realistica;
            \item Per lo stream processing: Apache Kafka;
            \item Per lo storage dei dati: ClickHouse (database colonnare);
            \item Per il data visualization: Grafana.
        \end{itemize}
    \subsection{Considerazioni}
    Interessante applicazione IoT, la possibilità di modellare dei dati rappresentandoli
    in un ambiente che mostra lo stato di salute di una città è appassionante e
    in linea con i bisogni moderni degli utenti sempre più attenti alla salute.
    I requisiti sono chiari e ben esposti e lo stack tecnologico ben definito ma
    comunque flessibile, il capitolato è in generale completo ed esaustivo.

    \section{Valutazione C7 -- ChatGPT vs BedRock developer analysis}
    \subsection{Descrizione}
        \begin{itemize}
            \item Proponente:
            \begin{itemize}
                \item Zero12.
            \end{itemize}

            \item Committente:
            \begin{itemize}
                \item Prof. Tullio Vardanega e Prof. Riccardo Cardin.
            \end{itemize}

            \item Obbiettivo:
            \begin{itemize}
                \item Utilizzare il machine learning per produrre epic e user stories automaticamente.
            \end{itemize}

        \end{itemize}
    L'obiettivo prefissato dal capitolato è la realizzazione di un applicativo in grado di analizzare i requisiti di business e il codice
    sorgente al fine di produrre epic\&user stories, mediante l'utilizzo di sistemi come ChatGPT e AWS BedRock.
    Il capitolato pertanto si concentra sull'utilizzo di sistemi di comprensione e analisi testuali sfruttando le capacità dell'emergente
    tecnologia dell'intelligenza artificiale.
    Una volta generate le epic\&user storie, la verifica di correttezza e completezza dovrà avvenire manualmente tramite la webapp, dando la possibilità
    all'utente di visualizzarle e valutarle.

    Il capitolato prevede lo studio e l'utilizzo di diverse tecnologie per ottenere:
    \begin{itemize}
        \item Middleware in grado di ricevere in input requisiti di business e codice sorgente per la produzione di epic\&user stories;
        \item Plugin per VisualStudio Code;
        \item Plugin per Apple XCode;
        \item Sviluppo modulare dell'applicativo in modo da poter confrontare il sistema utilizzando ChatGPT rispetto a AWS BedRock;
        \item Architettura basata su micro-servizi.
    \end{itemize}

    \subsection{Dominio Tecnologico}
    Tecnologie consigliate:
        \begin{itemize}
            \item Gestione del container: AWS Fargate;
            \item Per lo storage dei dati: MongoDB;
            \item Sviluppo di API: NodeJS;
            \item Linguaggio per lo sviluppo del plugin per XCode: Python;
            \item Linguaggio per lo sviluppo del plugin per VisualStudio Code: Typescript.
        \end{itemize}
    \subsection{Conclusioni}
    Si tratta di un prodotto dedicato agli sviluppatori, difficile da testare il grado di correttezza dell'output, XCode richiede MacOS (ma Zero12 fornirebbe computer nella loro sede), forte carattere esplorativo, sarebbe un primo approccio ad AWS (standard di settore), crediti AWS inclusi.

\section{Valutazione C8 -- JMAP: il nuovo protocollo per la posta elettronica}
    \subsection{Descrizione}
        \begin{itemize}
            \item Proponente:
            \begin{itemize}
                \item Zextras.
            \end{itemize}
            \item Committente:
            \begin{itemize}
                \item Prof. Tullio Vardanega e Prof.
                Riccardo Cardin.
            \end{itemize}
            \item Obbiettivo:
            \begin{itemize}
                \item Utilizzare il protocollo JMAP per realizzare un'applicazione per lo scambio di email.
            \end{itemize}
        \end{itemize}
        Il capitolato si propone di esplorare nuovi sviluppi nella comunicazione e-mail, probabilmente la tecnologia di comunicazione più utilizzata al mondo.
        Per farlo bisogna lavorare con il protocollo JMAP che dovrà sostituire il protocollo IMAP precedentemente utilizzato per queste applicazioni.
        Scopo di tale lavoro è, da parte del proponente, capire se ha senso investire tempo e denaro per estendere questo standard in Carbonio, un servizio di
        collaborazione che offre un’insieme di funzionalità con focus
        principale sulla gestione delle email.\\
        È previsto lo sviluppo di un servizio che permetta:
        \begin{itemize}
        \item Invio e ricezione di mail;
        \item Gestione, eliminazione, condivisione di una cartella;
        \item Gestione dei contenuti delle cartelle.
        \end{itemize}
        Opzionalmente anche l'implementazione di un sistema di sincronizzazione che permetta ad
        un client di mantenersi aggiornato con gli ultimi aggiornamenti della casella di posta visualizzata,
        contenente anche Calendari, Rubriche contatti, contatti e appuntamenti. Altri vincoli da rispettare:
        \begin{itemize}
        \item Il servizio sviluppato deve essere eseguibile in un sistema container;
        \item Il servizio sviluppato deve essere scalabile mediante l’inizializzazione di più nodi stateless;
        \item Opzionalmente aggiungere stress test che riescano a misurare le performance della soluzione.
        \end{itemize}
    \subsection{Dominio Tecnologico}
        Tecnologie consigliate:
        \begin{itemize}
            \item Linguaggio di programmazione: Java;
            \item Librerie per l’implementazione del protocollo JMAP: iNPUTmice o jmap;
            \item Sistema per gestione container: Docker.
        \end{itemize}
    \subsection{Conclusioni}
        Capitolato di carattere esplorativo basato su standard recenti, è un lavoro basato sul protocollo e non su applicazione, il quale tenta di innovare il servizio email.
        La documentazione fornita è estensiva e il supporto è fornito direttamente dal development team di Carbonio.
        È interessante la possibilità di valutare le performance tramite stress test.

\section{Valutazione C9 -- ChatSQL: creare frasi SQL da linguaggio naturale}
    \subsection{Presentazione}
        \begin{itemize}
            \item Proponente:
            \begin{itemize}
                \item Zucchetti.
            \end{itemize}
            \item Committente:
            \begin{itemize}
                \item Prof. Tullio Vardanega e Prof.
                Riccardo Cardin.
            \end{itemize}
            \item Obbiettivo:
            \begin{itemize}
                \item Sviluppare un'applicazione che permetta di generare prompt di testo utili alla creazione di comandi SQL.
            \end{itemize}
        \end{itemize}
    \subsection{Descrizione del capitolato}
        Il proponente chiede di sviluppare un'applicazione che permetta di generare, immettendo una richiesta in linguaggio naturale e il database (o parte di esso) in un modello di machine learning adeguatamente istruito, una frase anch'essa in linguaggio naturale, la quale attraverso l'utilizzo di ChatGPT possa generare i comandi SQL richiesti.
        L'applicazione deve svolgere quindi i seguenti compiti:
        \begin{itemize}
            \item  Archiviazione della descrizione della struttura di un database, possibilmente commentata in
            tutte le sue parti;
            \item  Maschera di richiesta di una frase di interrogazione del database in linguaggio naturale;
            \item  Procedura che combina la richiesta di interrogazione con le informazioni della struttura del
            database creando un “prompt” che sottoposto ad un sistema di AI fornisce l’interrogazione
            equivalente al linguaggio naturale in linguaggio SQL.
        \end{itemize}

    \subsection{Dominio Tecnologico}
    Il proponente non pone vincoli sulle tecnologie da utilizzare, ma suggerisce, riguardo i Large Language Model, l'utilizzo di ChatGPT, Palm o LLaMa.

    \subsection{Conclusioni}
        Sono emersi dubbi su come valutare i prompt generati e ci si è soffermati su come elaborare il dataset per ricavare le informazioni necessarie alla creazione del prompt.
        Interessante l'utilizzo di tecnologie attuali come le intelligenze artificiali e la possibilità di fare training di modelli di machine learning, in questo aiuta la flessibilità nella scelta dei modelli di Large Language Model a supporto.

\end{document}