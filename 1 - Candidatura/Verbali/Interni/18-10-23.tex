\documentclass[12pt,a4paper]{article}
\usepackage[italian, english]{babel}
\usepackage[utf8]{inputenc}
\usepackage[T1]{fontenc}
\usepackage{amsmath}
\usepackage{amsfonts}
\usepackage{xurl}
\usepackage{graphicx}
\usepackage[dvipsnames]{xcolor}
\usepackage{tikz}\usepackage{multirow}
\usepackage{comment}
\usepackage{tabularx}
\usepackage{multirow}

\makeatletter
\newcommand*{\rom}[1]{\expandafter\@slowromancap\romannumeral #1@}
\makeatother

%=====================INIZIO DOCUMENTO=====================
\begin{document}

%%%%%%%%%%%%%%%% header %%%%%%%%%%%%%%%%%%%%%%%%%%%%%%

\noindent\begin{minipage}{0.3\textwidth}
    \includegraphics[width=\linewidth]{logo.png}
\end{minipage}%
\hfill%
\begin{minipage}{0.6\textwidth}\raggedright
    \huge
    ERROR\_418\\
    Verbale 18/10/23
\end{minipage}

%%%%%%%%%%%%%%%% Sezione informativa %%%%%%%%%%%%%%%%%%
\large
\setlength{\extrarowheight}{9pt}
\raggedright
\begin{tabularx}{0.9\textwidth} [right] {
        >{\raggedright\arraybackslash}X
        >{\raggedright\arraybackslash}X
    }
    Mail:           & error418swe@gmail.com                              \\
    Redattori:      & Antonio Oseliero, Alessio Banzato                  \\
    Verificatori:   & Riccardo Carraro, Giovanni Gardin, Rosario Zaccone \\
    Amministratori: & Silvio Nardo, Mattia Todesco                       \\
    Destinatari:    & T. Vardanega, R. Cardin
\end{tabularx}
%%%%%%%%%%%%%%%%%%%%%% Presenze %%%%%%%%%%%%%%%%%%%%%%%%
\vspace{3mm}\hline\hline
\raggedright
\begin{tabular}{c c}
    \multicolumn{2}{c}{Inizio Meeting: 15:00 \hspace{4mm}
    Fine Meeting: 16:00 \hspace{4mm} Durata:1:00h} \\
    Presenze: &                                    \\
\end{tabular}

\begin{center}
    \begin{tabular}{ |c|c|c|c|c| }
        \hline
        Nome     & Durata Presenza &  & Nome     & Durata Presenza \\
        \hline
        Antonio  & 1:00h           &  & Alessio  & 1:00h           \\
        \hline
        Riccardo & 1:00h           &  & Giovanni & 1:00h           \\
        \hline
        Rosario  & 1:00h           &  & Silvio   & 1:00h           \\
        \hline
        Mattia   & 1:00h           &  &          &                 \\
        \hline

    \end{tabular}
\end{center}

\newpage

Ordine del giorno:
\begin{itemize}
    \item Discussione capitolati;
    \item Discussione proposte per nomi e loghi.
\end{itemize}

\section{Capitolati}
Si sono scelti, in ordine di preferenza: C9, C5, C3. La tabella di coordinamento è stato aggiornata di conseguenza.

\subsection{C3}
Valutazione iniziale: capitolato molto lungo e ricco di richieste.
Domande al proponente:
\begin{itemize}
    \item Ancora da formulare.
\end{itemize}

\subsection{C3}
Valutazione iniziale: il capitolato richiede l'implementazione di un numero limitato di feature. La difficoltà principale individuata è la gestione del 3D.\\
Domande al proponente:
\begin{itemize}
    \item Chiarimenti in merito alla figura dell'amministratore;
    \item Controllo del movimento di veicoli all'interno del magazzino (controllo della congestione);
    \item Chiarimenti in merito all'obiettivo minimo 3: \\
    Possibilità di selezionare un prodotto (oggetto all'interno della scaffalatura) e richiederne lo spostamento in un'altra area (altra scaffalatura o la stessa);
    \item Domanda in merito alla necessità di Hardware necessario alla modellazione 3D ed eventuale fornitura da parte dell'azienda.
\end{itemize}

\subsection{C9}
Valutazione iniziale: il capitolato richiede una forte comprensione del "prompt engineering" al fine di generare mediante un modello un prompt ad-hoc per chat-GPT (o altri modelli), con obiettivo finale l'interrogazione di una base di dati.\\
Domande al proponente:
\begin{itemize}
    \item Fornitura di Hardware necessario al training e alla creazione del modello;
    \item Come creare i prompt intermedi.
\end{itemize}

\section{Coordinamento}
\subsection{E-mail}
Mail/Riflettore per il gruppo creata e impostato l'inoltro automatico delle mail da quelle del gruppo alle mail individuali.

\subsection{Repository}
Creata l'organizzazione GitHub e aggiunti tutti i membri.

\subsection{Risoluzioni}
Confermare l'inoltro automatico nella propria casella di posta @studenti.unipd.it.\\
Giovanni contatta i proponenti C5 e C9 per esporre le domande proposte.\\
Antonio si occupa di impostare le pipeline per la creazione dei documenti.\\
Fissare meeting il prima possibile non appena ricevuta risposta dai proponenti.\\
Darsi delle regole per il WoW su commit, gestione PR, CI e strumenti adottati.

\end{document}