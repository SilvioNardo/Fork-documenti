\documentclass[12pt,a4paper]{article}
\usepackage[italian, english]{babel}
\usepackage[utf8]{inputenc}
\usepackage[T1]{fontenc}
\usepackage{amsmath}
\usepackage{amsfonts}
\usepackage{xurl}
\usepackage{graphicx}
\usepackage[dvipsnames]{xcolor}
\usepackage{tikz}\usepackage{multirow}
\usepackage{comment}
\usepackage{tabularx}
\usepackage{multirow}

\makeatletter
\newcommand*{\rom}[1]{\expandafter\@slowromancap\romannumeral #1@}
\makeatother

%=====================INIZIO DOCUMENTO=====================
\begin{document}

%%%%%%%%%%%%%%%% header %%%%%%%%%%%%%%%%%%%%%%%%%%%%%%

\noindent\begin{minipage}{0.3\textwidth}
    \includegraphics[width=\linewidth]{logo.png}
\end{minipage}%
\hfill%
\begin{minipage}{0.6\textwidth}\raggedright
    \huge
    ERROR\_418\\
    Verbale 26/10/23
\end{minipage}

%%%%%%%%%%%%%%%% Sezione informativa %%%%%%%%%%%%%%%%%%
\large
\setlength{\extrarowheight}{9pt}
\raggedright
\begin{tabularx}{0.9\textwidth} [right] {
        >{\raggedright\arraybackslash}X
        >{\raggedright\arraybackslash}X
    }
    Mail:           & error418swe@gmail.com                              \\
    Redattori:      & Antonio Oseliero, Alessio Banzato                  \\
    Verificatori:   & Riccardo Carraro, Giovanni Gardin, Rosario Zaccone \\
    Amministratori: & Silvio Nardo, Mattia Todesco                       \\
    Destinatari:    & T. Vardanega, R. Cardin
\end{tabularx}
%%%%%%%%%%%%%%%%%%%%%% Presenze %%%%%%%%%%%%%%%%%%%%%%%%
\vspace{3mm}\hline\hline
\raggedright
\begin{tabular}{c c}
    \multicolumn{2}{c}{Inizio Meeting: 18:00 \hspace{4mm}
    Fine Meeting: 18:20 \hspace{4mm} Durata:20min} \\
    Presenze: &                                    \\
\end{tabular}

\begin{center}
    \begin{tabular}{ |c|c|c|c|c| }
        \hline
        Nome     & Durata Presenza &  & Nome     & Durata Presenza \\
        \hline
        Antonio  & 20min          &  & Alessio  & 20min           \\
        \hline
        Riccardo & 20min           &  & Giovanni & 20min           \\
        \hline
        Rosario  & /           &  & Silvio   & 20min           \\
        \hline
        Mattia   & 20min           &  &SanMarco Informatica          & 20min                \\
        \hline

    \end{tabular}
\end{center}

\newpage



\section{Domande riguardanti il capitolato con SanMarco Informatica}
Dopo esserci presentati abbiamo chiesto maggiorni informazioni all'azienda SanMarco Informatica riguardo a dubbi che avevamo sul capitolato C5: Warehouse management 3D.
Abbiamo chiarito che:
\begin{itemize}
    \item quando si parla di notifiche di trasferimento si intende chiamate REST API
    \item i bin sono aree generiche dove mettere i materiali (nel pavimento, sui mezzi,sugli scaffali,...)
    \item uno scaffale può avere bin di diverse dimensioni. Gli scaffali sono divisi in ripiani e i ripiani in bin. In uno stesso scaffale possono coesistere bin di dimensioni diverse (su ripiani diversi).
    I bin hanno dimensioni costanti.
    \item controllare se una merce è compatibile con un bin (per posizione o per merci limitrofe) è una feature interessante ma non richiesta dal capitolato
    \item dobbiamo gestire un database relazionale il quale dovrà essere precaricato nel programma, il quale non ha memoria quindi non modifica il database e se la pagina viene ricaricata, il programma riparte dallo stato di partenza
    \item le tecnologie consigliate per la gestione del database sono: PostgreSQL, MariaDB e SQL Server
    \item poter creare, modificare e spostare scaffali manualmente è una feature utile ma non è richiesto che lo stato del magazzino venga salvato in memoria quindi si tratta di funzionalità supplementari
\end{itemize} 
\section{Altri argomenti affrontati}
\begin{itemize}
    \item abbiamo concordato con l'azienda la frequenza dei meeting: almeno una volta al mese ma se si ritiene utile si possono richiedere altri incontri. L'azienda si è inoltre resa disponibile per la ricezione di qualsiasi email ritenessimo necessaria
    \item In fine ci è stato chiesto il motivo che ci ha spinto a sceglere il loro capitolato e, in maniera unanime, abbiamo concordato che lavorare con il 3D è molto stimolante e interessante anche a livello lavorativo per la sua ampia l'utilità
\end{itemize} 
\end{document}
