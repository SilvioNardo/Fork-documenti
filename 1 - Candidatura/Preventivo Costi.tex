\documentclass[a4paper, twoside]{article}
%% Language and font encodings
\usepackage[english]{babel}
\usepackage[utf8x]{inputenc}
\usepackage[T1]{fontenc}
%% Sets page size and margins
\usepackage[a4paper,top=3cm,bottom=2cm,left=3cm,right=3cm,marginparwidth=1cm]{geometry}
%% Useful packages
\usepackage{amsmath}
\usepackage{graphicx}
\usepackage[colorinlistoftodos]{todonotes}
\usepackage[colorlinks=false, allcolors=black]{hyperref}
\usepackage{tabularx}
\usepackage{booktabs}
\usepackage{caption}
%------------------------
\title{\Huge Preventivo Costi}
\author{Error\_418}
\newcommand{\HRule}{\rule{\linewidth}{0.5mm}}
\setlength {\marginparwidth }{2cm}
%%page number
\pagestyle{plain}
%---------------
\begin{document}
\sffamily
\begin{titlepage}
%----------------------------------------------------------------------------------------
%	LOGO SECTION
%----------------------------------------------------------------------------------------
\centering
\includegraphics[width=8cm]{logo.png}\\[1.5cm]
%----------------------------------------------------------------------------------------
\center % Center everything on the page
%----------------------------------------------------------------------------------------
%	HEADING SECTIONS
%----------------------------------------------------------------------------------------
\textsf{\Large ERROR\_418}\\[0.5cm]
\textsf{\Large DOCUMENTAZIONE PROGETTO}\\[0.5cm]
%----------------------------------------------------------------------------------------
%	TITLE SECTION
%----------------------------------------------------------------------------------------
\makeatletter
\HRule \\[0.4cm]
{ \huge \bfseries \@title}\\[0.4cm]
\HRule \\[1.5cm]
%----------------------------------------------------------------------------------------
%	AUTHOR SECTION
%----------------------------------------------------------------------------------------

\begin{center} % Centro delle sezioni Redattore e Validatori
    \Large
    \setlength{\extrarowheight}{9pt}
    \raggedright
    \begin{tabularx}{0.9\textwidth} [right] {
            >{\raggedright\arraybackslash}X
            >{\raggedright\arraybackslash}X
        }
        Mail:           & error418swe@gmail.com                              \\
        Redattori:      & Antonio Oseliero, Alessio Banzato                  \\
        Verificatori:   & Riccardo Carraro, Giovanni Gardin, Rosario Zaccone \\
        Amministratori: & Silvio Nardo, Mattia Todesco                       \\
        Destinatari:    & T. Vardanega, R. Cardin
    \end{tabularx}
\end{center}

\vfill % Fill the rest of the page with whitespace
\end{titlepage}
\large

\newpage % Vai alla pagina successiva

%%%%%%%%%%%%%%% SEZIONE CONTENUTO %%%%%%%%%%%%%%%%%%%%%%%%%%%%%%%%%%%%%


\section{Analisi preliminare}
\large{A seguito di una quanto più accurata analisi relativa al carico di lavoro e al monte ore necessario al suo svolgimento, il gruppo Error\_418 ha individuato tre periodi di sviluppo fondamentali, quali:}
\begin{itemize}
    \item Periodo raccolta e analisi dei requisiti;
    \item Periodo di sviluppo della Requirements and Technologies Baseline;
    \item Periodo di sviluppo del Minimun Viable Product (MVP).
\end{itemize}

\subsection{Periodo raccolta e analisi dei requisiti}
In questo periodo, il ruolo centrale è svolto dagli analisti, che mediante un costante e produttivo rapporto con l'azienda delineeranno i requisiti che il prodotto finale dovrà possedere, soffermandosi nel carpirne dettagli, delimitazioni e standard di accetazione.

\subsection{Periodo di sviluppo della Requirements and Technologies Baseline}
A seguito del risultato prodotto nella fase precedente di analisi dei requisiti, ampio spazio sarà dedicato alla progettazione e all'utilizzo pratico delle tecnologie necessarie allo sviluppo, dimostrando la loro validità e funzionalità. \\
Il prodotto finale di questa fase sarà il Proof of Concept dimostrante la concreta possibilità di poter utilizzare le tecnologie in modo cooperante tra loro.

\subsection{Periodo di sviluppo del Minimun Viable Product (MVP)}
Dimostrata la concreta possibilità implementativa mediante il Proof of Concept, il gruppo impiegherà le risorse nella realizzazione materiale di un prodotto software atto al soddisfacimento \textit{almeno} dei requisiti minimi di accettazione, fornendo tutte le funzionalità richieste per essere considerato valido, ponendo attenzione non solo al lato funzionale ma anche, e soprattutto, al lato implementativo.

\subsection{Amministratori, Responsabili e Verificatori}
Amministratori, responsabili e verificatori sono ruoli che accompagneranno l'intero sviluppo, presenti in ogni periodo precedentemente descritto e con responsabilità ben definite. Il loro ruolo sarà delineare le linee di sviluppo che il gruppo dovrà mantenere durante l'avanzamento del progetto, il rispetto delle norme del way of working e verificare che quanto prodotto sia conforme nei confronti di quanto pianificato e di quanto richiesto.

\section{Preventivo dei Costi}
\begin{center}
    \begin{tabular}{ |c|c|c|c|c|c|c|c| }
        \hline
        \textbf{Membro} & \textbf{Resp.} & \textbf{Ammin.} & \textbf{Analista} & \textbf{Prog.} & \textbf{Program.} & \textbf{Verifi.} & \textbf{Totale} \\
        \hline
        Alessio & 15 & 10 & 10 & 14 & 28 & 18 & 95 \\
        \hline
        Antonio & 15 & 10 & 10 & 14 & 28 & 18 & 95 \\
        \hline
        Giovanni & 15 & 10 & 10 & 14 & 28 & 18 & 95 \\
        \hline
        Mattia & 15 & 10 & 10 & 14 & 28 & 18 & 95 \\
        \hline
        Riccardo & 15 & 10 & 10 & 14 & 28 & 18 & 95 \\
        \hline
        Rosario & 15 & 10 & 10 & 14 & 28 & 18 & 95 \\
        \hline
        Silvio & 15 & 10 & 10 & 14 & 28 & 18 & 95 \\
        \hline
        \hline
        \textbf{Totale Ore} & 105 & 70 & 70 & 98 & 196 & 126 & 665 \\
        \hline
        \textbf{Costo Orario} (€) & 30,00 & 20,00 & 25,00 & 15,00 & 15,00 & 30,00 &  \\
        \hline
        \textbf{Costo Ruolo} (€) & 3.150,00 & 1.400,00 & 1.720,00 & 2.450,00 & 2.940,00 & 1.890,00 & \\
        \hline
        \multicolumn{8}{|c|}{ \textbf{Totale preventivato: € 13.580,00}} \\
        \hline
    \end{tabular}
\end{center} 

\vspace{1cm}
\noindent
Cordiali saluti,\\
\textit{Error\_418} - Gruppo 7 di Ingegneria del Software
\end{document}