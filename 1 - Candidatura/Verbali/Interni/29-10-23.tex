\documentclass[12pt,a4paper]{article}
\usepackage[italian, english]{babel}
\usepackage[utf8]{inputenc}
\usepackage[T1]{fontenc}
\usepackage{amsmath}
\usepackage{amsfonts}
\usepackage{xurl}
\usepackage{graphicx}
\usepackage[dvipsnames]{xcolor}
\usepackage{tikz}\usepackage{multirow}
\usepackage{comment}
\usepackage{tabularx}
\usepackage{multirow}

\makeatletter
\newcommand*{\rom}[1]{\expandafter\@slowromancap\romannumeral #1@}
\makeatother

%=====================INIZIO DOCUMENTO=====================
\begin{document}

%%%%%%%%%%%%%%%% header %%%%%%%%%%%%%%%%%%%%%%%%%%%%%%

\noindent\begin{minipage}{0.3\textwidth}
    \includegraphics[width=\linewidth]{logo.png}
\end{minipage}%
\hfill%
\begin{minipage}{0.6\textwidth}\raggedright
    \huge
    ERROR\_418\\
    Verbale 29/10/23
\end{minipage}

%%%%%%%%%%%%%%%% Sezione informativa %%%%%%%%%%%%%%%%%%
\large
\setlength{\extrarowheight}{9pt}
\raggedright
\begin{tabularx}{0.9\textwidth} [right] {
        >{\raggedright\arraybackslash}X
        >{\raggedright\arraybackslash}X
    }
    Mail:           & error418swe@gmail.com                              \\
    Redattori:      & Antonio Oseliero, Alessio Banzato                  \\
    Verificatori:   & Riccardo Carraro, Giovanni Gardin, Rosario Zaccone \\
    Amministratori: & Silvio Nardo, Mattia Todesco                       \\
    Destinatari:    & T. Vardanega, R. Cardin
\end{tabularx}
%%%%%%%%%%%%%%%%%%%%%% Presenze %%%%%%%%%%%%%%%%%%%%%%%%
\vspace{3mm}\hline\hline
\raggedright
\begin{tabular}{c c}
    \multicolumn{2}{c}{Inizio Meeting: 15:00 \hspace{4mm}
    Fine Meeting: 16:30 \hspace{4mm} Durata: 1:30h} \\
    Presenze: &                                    \\
\end{tabular}

\begin{center}
    \begin{tabular}{ |c|c|c|c|c| }
        \hline
        Nome     & Durata Presenza \\
        \hline
        Antonio  & 1:30h           \\
        \hline
        Alessio & 1:30h           \\
        \hline
        Rosario  & 1:30h           \\
        \hline
        Giovanni  & 1:30h           \\
        \hline

    \end{tabular}
\end{center}

\newpage

Ordine del giorno:
\begin{itemize}
    \item panoramica dei comandi base di git;
    \item panoramica delle funzioni di collaborazione di GitHub;
    \item discussione sulle strategie di branching da adottare.
\end{itemize}

\section{git e Github}
Si è effettuata una panoramica sui comandi git di base come init, add, push, pull, status, branch, checkout e sulle funzioni di pull request di GitHub.

\section{Strategie di branching}
Si sono confrontante due strategie di branching:
\begin{itemize}
    \item \textbf{main-feature}: un ramo mainline e feature branch atomici e indipendenti;
    \item \textbf{main-dev-feature}: un ramo mainline, un ramo di sviluppo e feature branch atomici e indipendenti.
\end{itemize}

Dalla discussione è emerso che la strategia 1 è da preferirsi per progetti rilasciati in modalità rolling, senza versionamento. Tuttavia, poiché il dominio d'applicazione del capitolato è on-premise, appare più adatto un approccio con versionamento (strategia 2).
Inoltre, il proponente potrebbe avere la necessità di fornire versioni del software diverse.
Per concludere, lo stack tecnologico che sarà verosimilmente adottato avrà numerose dipendenze verso librerie esterne, che dovranno essere congelate prima del rilascio.\\
Per questi motivi, si è scelto di approcciare lo sviluppo seguendo la strategia 2. Non si tratta comunque di una scelta vincolante perché sarà sempre possibile commutare la strategia.\\
Si sottolinea comunque l'enorme importanza di progettare feature piccole, ben delineate e atomiche.
\end{document}

